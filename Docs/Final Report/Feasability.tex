\documentclass{report}
\usepackage[pdftex]{graphicx}
\usepackage[english]{babel}
\usepackage{float}
\begin{document}
\title{Temperature and Humidity Monitoring System}
\author{Marcel Englmaier, Justin Koehler, and Jason Pearson}
\maketitle
\tableofcontents
\newpage

\subsection*{Background}
\addcontentsline{toc}{subsection}{Background}
Our client has many needs that have been unmet for various reasons. Existing problems make the current situation work improperly. Our project provides solutions to those needs and provides resolutions to their problems.
\newline
\indent
Per Wikipedia, Western Michigan University's (WMU for short) Parkview campus was built in 2003 at a cost of $\$$72.5 million and is the home of the Western Michigan University College of Engineering and Applied Sciences (CEAS for short). WMU’s engineering website explains that WMU has “state-of-the-art resources housed in a $\$$100 million high-tech facility”. Sadly, our client has advised that this did not include any automated temperature or humidity sensors and reporting equipment in any of the rooms. 
These are absolutely critical in rooms that maintain computer, technology, manufacturing, and scientific equipment to safeguard the investment and resources of the university.  There are many risks that computer equipment face as they spend their entire life conducting electricity and being made of rust-prone metals. Standard servers are recommended to be kept at an average temperature of $25\,^{\circ}{\rm C}$ or less, with automatic shut-off or critical shutdown temperature maximums of $35\,^{\circ}{\rm C}$. They must also be kept dry as any condensation will not only short any circuit boards it touches, but cause the servers themselves to rust, as well as the metal racks that support them. Aside from rusts and shorts, excess humidity is a cause of molding and mildewing which is unhealthy for personnel and students, and also damages hardware and clogs air filters.
\newline
\indent
There are many factors that provoke the need for this monitoring, from the downtime of a website, to the security networks that safeguard a campus, the need to have digital phones online, to safeguard data that would be lost in a failure, to the overall cost of the hardware. As an example, one server cluster with the moniker "Thor" has a hardware value of $\$$400,000 which would result in an excessive loss for the university and to research done by many departments if it were damaged.
\newline
\indent
Our client informed us that there have been several incidents where the temperature of servers increased unhindered to the point that equipment was destroyed due to this lack of automated environment reporting. One such incident where the temperature increased without staff knowing resulted in a ~$\$$500,000 loss. A previous loss due to humidity occurred when the humidity rose to the point of condensation and large steel paper rolls accrued a layer of surface rust, rendering it useless. It needed to be replaced, causing monetary damages and downtime.
\newline
\indent
Since these fateful incidents, WMU has had students implement several forms of reporting, and currently uses the a device called "Temperature@lert WiFi Edition" to keep track of the temperature of rooms around campus. These sensors work very well but their major flaw is that they are very expensive. These sensors cost upwards of three hundred dollars per sensor and have many features that are useful, but unnecessary for our purposes. These devices also lack a very important feature: a central management system to view all the sensors. To alleviate this problem we proposed to create a web site that would communicate with a network of reliable, home brewed sensor units. This website was originally created by another WMU Computer Science Senior Design team and it currently is used to communicate with the Temperature@lert sensors.
\newline
\indent  
We are unaware at this time of some specific information regarding the facilities such as the type and rating of their heating, ventilation, and air conditioning systems (HVAC for short), the dollar value of all the equipment lost in the past, the British Thermal Units (BTU’s for short) that are generated by this equipment, or how fast the temperature would increase in the rare event of an HVAC malfunction. It is clear that WMU have a need for automated reporting, and our solution will be more than adequate regardless of this information. Our client has provided us with some basic information, which shows that due to the thermal mass of the equipment in the rooms, a notification within several minutes would be more than adequate to prevent damage. As our prototype currently stands, there is roughly up to a 3 minute delay before a notification would be sent due to a sixty second temperature fetch cycle from the raspberry pi, a sixty second fetch cycle by the web server, and a sixty second processing cycle that generates the web pages, generates reports, processes data to the database, and would send an alert if the circumstances arose. This could easily be reduced to a total of sixty seconds for the whole process, and very well may be user-selectable on the web page at our client's request. We opted for a slower cycle to reduce server load and resource consumption. Our solution will increase autonomy, provide reporting, reduce cost, add better functionality, and provide a product that can be used by students and administrators alike.
\newline
\indent
The currently used Temperature@lert WiFi sensor has accuracy of $\pm$0.$5\,^{\circ}{\rm C}$. The maximum and minimum temperatures that the current sensor will calculate are -$10\,^{\circ}{\rm C}$ and +$85\,^{\circ}{\rm C}$. The current sensor also gives humidity readings. This is not high priority, but if we are able to implement it that would be desirable. The humidity readings that the current sensors give are between 10$\%$ and 90$\%$ relative humidity. This relative humidity has $\pm$3$\%$ relative humidity accuracy. One major feature of the sensor is the fact that it can be used over the network using wired or wireless connections. The wireless specifications that it abides by are the 802.11b/g standards and allow for WPA/WEP security. These are the features that are used by the system that we need to implement on our client devices.
\newpage

\begin{figure}[H]
	\makebox[\textwidth]{\includegraphics[scale=0.5]{Website.PNG}}
	\caption{Initial View Of Site}
\end{figure}
\indent
The main page of the website that we inherited from the previous team is shown above. On this page there is a graph of all the temperatures of reported by all the sensor units that are currently being tracked. It does not currently display any humidity readings. We will be evolving the site to include this information in future releases. The existing framework alloys easy data viewing with little programming effort. We will use the existing framework and expand the code to include humidity. Additionally, we will be adding additional graphing functionality using the logged historical data.
\newpage


\indent
The login process is very basic at this time. The existing web site uses lackluster security protocols, minimal, data sanitization, and no input verification. Our client has asked us to upgrade all security features and implement a system that will prevent session hijacking, network eavesdropping, cross site scripting, and brute force attacks.

\begin{figure}[H]
	\makebox[\textwidth]{\includegraphics[scale=0.75]{LoginPage.PNG}}
	\caption{Login Page}
\end{figure}
\indent
As shown above, at this time the page has a simple “Username” and “Password” field along with a submit button. Our client has asked us to add additional features by adding a checkbox that disables the browser from remembering or saving this information. Our first “alpha” release will be using a test database with test users, test usernames, test passwords, and test data, so the security will not be an issue during this phase. When the user submits their credentials, the framework will reference the data to see if it matches what is in the database. If it does, the framework will provide further access to the site. If the authentication fails, the framework will require the user to try again. 
\newline
\indent
Our client asked us to create a single login page, but based on whether the user successfully authenticates as an administrator or a simple user provide the pages and views they have access to. The admin will have access to the identical pages as the user, but will have an administrator functionality added to the pages, which will allow them to add new users, rooms, and devices, as well as delete or modify existing users, rooms, and devices. A regular user will have a list of devices, whereas an administrator will see the same list but will have a button above said list that takes them to an “add device” page. The list will have an “edit” and “delete” button next to each device for administrators as well. The edit and delete pages will be similar to the add page, and will be basically the same for users, devices, and rooms.
\newpage

\begin{figure}[H]
	\makebox[\textwidth]{\includegraphics[scale=0.5]{AddPage.PNG}}
	\caption{Add Page}
\end{figure}
\indent
Upon logging in as a system administrator this is what the admin sees. This is the general hub for editing anything on the site. From here the admin can see rooms, users, room assignments and device types easily.
The page looks just the same for a room administrator when logging into the site. The only difference is that the room user won't have the option to edit any rooms, devices, etc. If a non logged-in user tries to access this page it redirects them to the login page so that all admin data isn't available to the public.
\newpage

\begin{figure}[H]
	\makebox[\textwidth]{\includegraphics[scale=0.75]{AddDevice.PNG}}
	\caption{Add Device To Network Page}
\end{figure}
\indent
The above figure is the form used when adding a new device to the network. The IP address is a major need in this form because it tells the server where to look for the data. The alert and critical thresholds are for setting the threshold for when to warn administrators for that room. The number of ports specifies simply the number of temperatures to expect coming from that device.
\newpage

\begin{figure}[H]
	\makebox[\textwidth]{\includegraphics[scale=0.75]{AddDeviceType.PNG}}
	\caption{Add Another Device Type}
\end{figure}
\indent
The above page is used to add a room to the monitoring system. The main purpose of a room is to group sensors together and make it easier to distribute work among the administrators.
\newline
\indent
The below figure shows the page where a new room can be added.
\begin{figure}[H]
	\makebox[\textwidth]{\includegraphics[scale=0.75]{AddRoom.PNG}}
	\caption{Adds a Room}
\end{figure}
\newpage

\indent
There are three different levels of users. The highest level of user is the main system administrator, or "master admin". This administrator is in control of the whole system. Their permissions include all the powers of the normal room administrator, but are able make and demote normal room administrators. 
\newline
\indent
The second tier of user is the normal room administrator. The room administrators are able see the stats of the rooms they have access and will receive alerts for those rooms. These administrators are able to add new users, add new devices, add new rooms, and change settings for these items.
\newline
\indent
The the third level of user is the regualar user. This user can see the graph of data on the front page and login. They also receive alerts, but are not able to change any settings.
\newline
\indent
There is a final type of user which requires no logging in. This option of user can only see the main page. It is beneficial to have it this way in case an administrator wants to quick check things and doesn't want to bother with logging in.
\newpage


\begin{figure}[H]
	\makebox[\textwidth]{\includegraphics[scale=0.75]{AddUser.PNG}}
	\caption{Page To Add Users}
\end{figure}
\indent
Above is the form for adding a new user. The required fields are name, email, and password. The password does have minimal requirements for good passwords. This is not very secure, and the email is also unverified.
\newpage

\begin{figure}[H]
	\makebox[\textwidth]{\includegraphics[scale=0.35]{ERDDiagram.png}}
	\caption{Entity Relationship Diagram}
\end{figure}
\indent
Above is the database diagram of the existing web site. We decided to not make minor changes because the database was designed in a manner that didn't allow humidity recording without adding another table
\newpage

\subsection*{Design Decisions}
\addcontentsline{toc}{subsection}{Design Decisions}

\newpage
\subsection*{Design}
\addcontentsline{toc}{subsection}{Design}

\newpage

\subsection*{Implementation}
\addcontentsline{toc}{subsection}{Implementation}
Now how did this design turn out. Before all the old site vital areas were shown. Now after implementing BootStrap we have a completely different front end.
Here is the initial view of the site once a user has logged in.
\begin{figure}[H]
	\makebox[\textwidth]{\includegraphics[scale=0.75]{firstlogin.PNG}}
	\caption{After Logging In}
\end{figure}
\newpage
Now when a user clicks on the tab it will expand to show what the current temperature and humidity are. Next to that is the status of the machine i.e. okay or critical. Upon mousing over the graph it will give the temperature or humidity at that time in history. See the next figure.
\begin{figure}[H]
	\makebox[\textwidth]{\includegraphics[scale=0.75]{opentabfirst.PNG}}
	\caption{Open Tab and Data}
\end{figure}
\newpage
These tabs are the same for each room and will show all appropriate data for each room. Next is the new and improved administration panel. 
On this panel there are five different tabs rooms, devices, users, room assignments and device types. All of these different devices the administrator is able to create, update and delete anything in those tabs. 
\begin{figure}[H]
	\makebox[\textwidth]{\includegraphics[scale=0.75]{adminpaneldesktop.PNG}}
	\caption{First View of Administration Panel}
\end{figure}
\newpage
The rooms tab has just those options and only has room name as its only parameter. See the next figure for the rooms user interface.
\begin{figure}[H]
	\makebox[\textwidth]{\includegraphics[scale=0.75]{roomsdesktop.PNG}}
	\caption{View of the Rooms Tab}
\end{figure}
\newpage
The next tab devices tab. This tab shows the device id, name, location, ip address, type of device, number of ports on the device, warning, alert and critical temperatures, and lastly status. The administrator can add, remove and update devices. 
\begin{figure}[H]
	\makebox[\textwidth]{\includegraphics[scale=0.75]{devicesdesktop.PNG}}
	\caption{Device Administration}
\end{figure}
\newpage
The users tab allows new users and administrators to be created as well as updating and deleting users. This tab also allows verification of the users email and phone alerts. For a verification the user receives a text or email with a verification code that that the user types in the code and this allows the system to verify that the text or email went through.
\begin{figure}[H]
	\makebox[\textwidth]{\includegraphics[scale=0.75]{usersdesktop.PNG}}
	\caption{User Administration}
\end{figure}
\newpage
The room assignments tab is very important. This is where administrators assign users to rooms. When a user is assigned to a room they will receive alerts when temperature benchmarks are reached.
\begin{figure}[H]
	\makebox[\textwidth]{\includegraphics[scale=0.75]{roomassignmentsdesktop.PNG}}
	\caption{User Room Assignments}
\end{figure}
\newpage
\subsection*{Testing}
\addcontentsline{toc}{subsection}{Testing}
There are two types of testing that we used in this project unit testing and functional testing. Unit Testing is a must because it will allow the user to test every feature when a new patch comes out and allows after each upgrade to check if the features still work. Unit tests cannot be used to check everything so there must also be functional tests that show the site is reacting correctly and can handle loads properly.
\newline
\indent
Unit tests for this project are ran through the command line and use laravel's testing framework. 
\newline
\indent
There were many different functional tests that we ran to test full functionality of our project. A few tests that we ran include
\begin {itemize}
\item E-mail Alerts
\item Text Message Alerts
\item 
\end {itemize}

\newpage
\subsection*{Security}
\addcontentsline{toc}{subsection}{Security}




\newpage
\subsection*{Maintenance}
\addcontentsline{toc}{subsection}{Maintenance}
After the completion of this project the current senior design team will offer no maintenance. All modifications will be done by the staff at Western Michigan University and anyone else who uses the project. This being said there are modifications that we had planned but didn't have time to get to that could be added to increase the user experience for the project. These include
\begin {itemize}
\item Adding a secure layer between the pi and the server
\item Adding historical data logging to the Raspberry Pi
\item Creating a way to retrieve historical data from the device if it was down for a period
\item Upgrading the user control area to have group administration
\item Making a program that would poll servers for their data and add it to a graph
\end {itemize}
\newpage

\subsection*{Resources}
\addcontentsline{toc}{subsection}{Maintenance}

\begin{itemize}
\item Raspberry Pi
\item Temperature Sensors
\item Humidity Sensors
\item Web Server
\item Soldering Equipment
\item Operating System loaded SD Cards with Raspbian
\item Power Connectors for Raspberry pi
\item WiFi Connectors for Raspberry pi
\item External Server to Host Website
\end{itemize}
\newpage


\subsection*{References}
\addcontentsline{toc}{subsection}{References}
For everything raspberry pi we use these sites
\begin{itemize}
\item http://www.raspberrypi.org/
\item http://www.raspbian.org/
\item C Programming 2nd Edition
\item http://www.adafruit.com/
\end{itemize}
For everything web server these are the sites we use
\begin{itemize}
\item http://laravel.com/docs/quick
\item http://www.w3schools.com/
\item http://www.noip.com
\item httpd:apache.org
\item http://www.w3.org
\item http://aws.amazon.com/
\end{itemize}
\newpage
\subsection*{Glossary}
\addcontentsline{toc}{subsection}{Glossary}
\begin{description}
\item [GUI] \hfill \\
Graphical User Interface. The windows a user interacts with.
\item [MSP430] \hfill \\
A 16-bit microcontroller platform made by Texas Instruments.
\item [Raspberry Pi] \hfill \\
 A credit-card-sized single-board computer developed by the Raspberry Pi Foundation.
\item [Arduino] \hfill \\
A series of microcontrollers that are very commonly used for computer to real world communications.
\item [Raspbian] \hfill \\
 A Debian based operating system that we will use for our Raspberry Pi’s
\item [CEAS] \hfill \\
College of Engineering and Applied Sciences at Western Michigan University.
\item [PHP] \hfill \\
A recursive acronym for “PHP Hypertext Preprocessor” - the web programming language being used.
\end{description}
\newpage
\subsection*{Ownership}
\addcontentsline{toc}{subsection}{Ownership}
\begin{description}
\item [Licenses] \hfill \\
Our project will be under several licenses. PHP is a free open source software released under the PHP License.
Laravel is licensed under the MIT license and per the agreement we are “free to modify, distribute and re-publish the source code on the condition that the copyright notices are left intact”.
In the event that our project was used to generate revenue, or be sold as a standalone software package, the license permits us “to incorporate Laravel into any commercial or closed source application”.  
The GNU license and will be open source and will be hosted for all to access and modify as they desire on GitHub.
\item [Intellectual Property (IP)] \hfill \\
As this project is being developed as a Senior Design project for Western Michigan University (WMU) at the direction of Dr. John Kapenga, WMU will retain the intellectual rights to the software.
\item [Non-Disclosure Agreement (NDA)] \hfill \\
No non-disclosure agreement is being used at this time.  The project is maintained on GitHub, which is freely and openly accessible to anyone who wishes to view it, and is thus tracked by search engines such as Google, where it is able to be searched for by anyone on the planet.
\item [Warranty] \hfill \\
A maintenance document has been created for the project. Other than that document and this document no other outside assistance is required by the members of this project.
\end{description}

\end{document}
